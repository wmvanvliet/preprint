\documentclass[a4paper, 10pt]{vanvliet_paper}
\input{acronyms}
\addbibresource{reading_models.bib}

\draft
\title{A large scale computational model of word recognition and its comparison with MEG data}

\author[1*]{Marijn van Vliet}
\author[1]{Oona Rinkinen}
\author[1]{Takao Shimizu}
\author[2]{Barry Devereux}
\author[1]{Riitta Salmelin}
\affil[1]{Department of Neuroscience and Biomedical Engineering, Aalto University}
\affil[2]{School of Electronics, Electrical Engineering and Computer Science, Queen's University Belfast}
\affil[*]{Corresponding author: marijn.vanvliet@aalto.fi}

\begin{document}
\maketitle

\begin{abstract}
\end{abstract}

\section{Introduction}

What computational steps is the brain performing when it recognizes some lines on a piece of paper as a specific word?
This question has been the focus of a large number of neuroimaging studies that examine brain activity during reading.
Noninvasive measurement techniques such as \gls{EEG}\cite{Grainger2009}, \gls{MEG}\cite{Salmelin2007} and \gls{fMRI}\cite{Price2012} have provided a wealth of information about when and where changes in activity might be expected during various tasks\cite{Carreiras2014}.
However, it is rarely straightforward to translate observations of brain activity into a mechanistic understanding of the computational process being performed by the brain\cite{Poeppel2012}.

The complex task to develop cognitive theories that describe how reading is performed by the brain is aided by the development of computational models.
These models are needed to make exact predictions from complex modern cognitive theories\cite{Taylor2012, Barber2007}.
However, for current models of reading, these predictions do not include estimations of actual neuroimaging data, and it is an often repeated sentiment that there should be more "contact" between the two\cite{Carreiras2014, Laszlo2012, Laszlo2014, Poeppel2012, Taylor2012}.

\section{Results}
\section{Discussion}
\section{Methods}
\section{Acknowledgements}
We acknowledge the computational resources provided by the Aalto Science-IT project.
This research was funded by the Academy of Finland (grant \#310988 to M.v.V, \#255349, \#256459, \#283071 and \#315553 to R.S.).
% TODO: Barry funding information

\newpage
\printbibliography{}

\end{document}


% Ok, dit gaat echt nergens meer over. Waar wil ik het over hebben?\
%
% Statement over neuroimaging en het modeleren van het proces daar achter.
% Voorbeelden van modellen: McClelland, DRP+, Laszlo & Plaut.
% Die laatste trekt een lijn naar neuroimaging: super cool.
% Deep learning schept nieuwe mogelijkheden: dezelfde stimuli kunnen nu gebruikt worden.
% Coole resultaten wat betreft object recognition.
% Dus in deze studie proberen we een deep learning model van reading te maken.
% En we vergelijken de activatie in het model met MEG activatie tijdens lezen.
