\documentclass[a4paper, 10pt]{vanvliet_paper}
\input{acronyms}
\addbibresource{reading_models.bib}

\draft
\title{A large scale computational model of word recognition and its comparison with MEG data}

\author[1*]{Marijn van Vliet}
\author[1]{Oona Rinkinen}
\author[1]{Takao Shimizu}
\author[2]{Barry Devereux}
\author[1]{Riitta Salmelin}
\affil[1]{Department of Neuroscience and Biomedical Engineering, Aalto University}
\affil[2]{School of Electronics, Electrical Engineering and Computer Science, Queen's University Belfast}
\affil[*]{Corresponding author: marijn.vanvliet@aalto.fi}

\begin{document}
\maketitle

\begin{abstract}
\end{abstract}

\section{Introduction}

What computational steps is the brain performing when it recognizes some lines on a piece of paper as a specific word?
This question has been the focus of a large number of neuroimaging studies that explore the brain activity generated during tasks that requires the participants to read, employing techniques such as \gls{EEG}\cite{Grainger2009}, \gls{MEG}\cite{Salmelin2007} and \gls{fMRI}\cite{Price2012}.
This has yielded a wealth of observations which c c

Understanding these computations strikes at the heart of both fundamental questions about human cognition and practical questions regarding cognitive disorders such as dyslexia, aphasia and Alzheimer's.
\section{Results}
\section{Discussion}
\section{Methods}
\section{Acknowledgements}
We acknowledge the computational resources provided by the Aalto Science-IT project.
This research was funded by the Academy of Finland (grant \#310988 to M.v.V, \#255349, \#256459, \#283071 and \#315553 to R.S.).
% TODO: Barry funding information

\newpage
\printbibliography{}

\end{document}
