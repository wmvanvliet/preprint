\documentclass[a4paper, 10pt]{vanvliet_paper}
\newacronym{FAS}{fas}{forward association strength}
\newacronym{BAS}{bas}{backward association strength}
\newacronym[\glslongpluralkey={event-related potentials}]{ERP}{erp}{event-related potential}
\newacronym{EEG}{eeg}{electroencephalography}
\newacronym{MEG}{meg}{magnetoencephalography}
\newacronym{EMG}{emg}{electromyography}
\newacronym{EOG}{eog}{electro-oculogram}
\newacronym{MR}{mr}{motor related}
\newacronym{MRP}{mrp}{motor related potential}
\newacronym{RT}{rt}{response time}
\newacronym{LSA}{lsa}{latent semantic analysis}
\newacronym{JAM}{jam}{judgment of associative memory}
\newacronym{BCI}{bci}{brain--computer interface}
\newacronym{LME}{lme}{linear mixed effects}
\newacronym{REML}{reml}{restricted maximum likelihood}
\newacronym{ML}{ml}{maximum likelihood}
\newacronym{ROC}{roc}{receiver operating characteristic}
\newacronym{AUC}{auc}{area under curve}
\newacronym{ROC-AUC}{roc-auc}{receiver operating characteristic -- area under curve}
\newacronym{NS}{ns}{negative slope}
\newacronym{LCMV}{lcmv}{linearly constrained minimum variance}
\newacronym{stLCMV}{\textnormal{st}lcmv}{spatio-temporal linearly constrained minimum variance}
\newacronym[\glslongpluralkey={components of interest}]{COI}{coi}{component of interest}
\newacronym[\glslongpluralkey={regions of interest}]{ROI}{roi}{region of interest}
\newacronym[\glslongpluralkey={generators of interest}]{GOI}{goi}{generator of interest}
\newacronym{ANOVA}{anova}{analysis of variance}
\newacronym{SNR}{snr}{signal-to-noise ratio}
\newacronym{lSVM}{\textnormal{l}svm}{linear support vector machine}
\newacronym{SVM}{svm}{support vector machine}
\newacronym{AoA}{a\textnormal{o}a}{age of acquisition}
\newacronym{PCA}{pca}{principal component analysis}
\newacronym{ICA}{ica}{independent component analysis}
\newacronym{MCMC}{mcmc}{Markov chain Monte Carlo}
\newacronym{LR}{lr}{logistic regression}
\newacronym{LDA}{lda}{linear discriminant analysis}
\newacronym{FC}{fc}{Fisher criterion}
\newacronym{SOA}{soa}{stimulus onset asynchrony}
\newacronym{IIR}{iir}{infinite impulse response}
\newacronym{RVM}{rvm}{random vector model}
\newacronym{RPM}{rpm}{random permutation model}
\newacronym{HAL}{hal}{hyperspace analogue to language}
\newacronym{CAR}{car}{common average reference}
\newacronym{MTG}{mtg}{middle temporal gyrus}
\newacronym{STG}{stg}{superior temporal gyrus}
\newacronym{MTC}{mtc}{medial temporal cortex}
\newacronym{ATC}{atc}{anterior temporal cortex}
\newacronym{AG}{ag}{angular gyrus}
\newacronym{IFG}{ifg}{inferior frontal gyrus}
\newacronym{fMRI}{\textnormal{f}mri}{functional magnetic resonance imaging}
\newacronym{CSP}{csp}{common spatial patterns}
\newacronym{GMM}{gmm}{Guassian mixture model}
\newacronym{OAS}{oas}{oracle approximating shrinkage}
\newacronym{bCSP}{\textnormal{b}csp}{bi-linear common spatial patterns}
\newacronym{HOOI}{hooi}{higher-order orthogonal iteration}
\newacronym{AIC}{aic}{Akaike information criterion}
\newacronym{GLM}{glm}{general linear model}
\newacronym{LARS}{lars}{least-angle regression}
\newacronym{BOLD}{bold}{blood-oxygen level dependant}
\newacronym{SWLDA}{swlda}{step-wise linear discriminant analysis}
\newacronym{OLS}{ols}{ordinary least squares}
\newacronym{SVR}{svr}{support vector regression}
\newacronym{RANSAC}{ransac}{random sample consensus}
\newacronym{MSE}{mse}{mean squared error}
\newacronym{L-BFGS-B}{l-bfgs-b}{Limited-memory Broyden--Fletcher--Goldfarb--Shanno with box constraints}
\newacronym{PDP}{pdp}{parallel distributed processing}
\newacronym{CNN}{cnn}{convolutional neural network}
\newacronym{ReLu}{r\textnormal{e}l\textnormal{u}}{rectified linear unit}
\newacronym{MNE}{mne}{minimum norm estimate}
\newacronym{ECD}{ecd}{equivalent current dipole}

\addbibresource{reading_models.bib}

\draft
\title{A large scale computational model of word recognition and its comparison with MEG data}

\author[1*]{Marijn van Vliet}
\author[1]{Oona Rinkinen}
\author[1]{Takao Shimizu}
\author[2]{Barry Devereux}
\author[1]{Riitta Salmelin}
\affil[1]{Department of Neuroscience and Biomedical Engineering, Aalto University}
\affil[2]{School of Electronics, Electrical Engineering and Computer Science, Queen's University Belfast}
\affil[*]{Corresponding author: marijn.vanvliet@aalto.fi}

\begin{document}
\maketitle

\begin{abstract}
\end{abstract}

\section{Introduction}

What computational steps is the brain performing when it recognizes some lines on a piece of paper as a specific word?
This question has been the focus of a large number of neuroimaging studies that examine brain activity during reading.
Noninvasive measurement techniques such as \gls{EEG}\cite{Grainger2009}, \gls{MEG}\cite{Salmelin2007} and \gls{fMRI}\cite{Price2012} have provided a wealth of information about when and where changes in activity might be expected during various tasks involving orthographic processing\cite{Carreiras2014}.
However, it is rarely straightforward to translate observations of brain activity into a mechanistic understanding of the computational process being performed by the brain\cite{Poeppel2012}.

Computational models facilitate the development of cognitive theories by allowing us to reason about conceptual "box and arrow" ideas in a qualitative and quantitative manner\cite{Barber2007, Price2018}.
However, the predictions made by existing models of reading are not directly comparible to actual neuroimaging data, and it is an often repeated sentiment that there should be more contact between the two\cite{Carreiras2014, Laszlo2012, Laszlo2014, Poeppel2012, Taylor2012}.

\section{Results}
\section{Discussion}
One may ask why the lexicon layer of the model is a one-hot encoded output vector.
This is plainly incompatible with how the brain works.
Some abstract semantical representation, such as word2vec or semantic features would clearly be better candidates.
The reason why the final layer of the model is the way it is, is because this is the point where a hard 90 degree turn needs to be made from orthographical similarity to semantic similarity.

The model used in this study is a standard convolutional design and has many shortcomings as a model of the brain.
Nevertheless, the fact that the model performs well despite these shortcomings shows the power of using deep learning models to implement cognitive theories.

\section{Methods}
\section{Acknowledgements}
We acknowledge the computational resources provided by the Aalto Science-IT project.
This research was funded by the Academy of Finland (grant \#310988 to M.v.V, \#255349, \#256459, \#283071 and \#315553 to R.S.).
% TODO: Barry funding information

\newpage
\printbibliography{}

\end{document}


% Ok, dit gaat echt nergens meer over. Waar wil ik het over hebben?\
%
% Statement over neuroimaging en het modeleren van het proces daar achter.
% Voorbeelden van modellen: McClelland, DRP+, Laszlo & Plaut.
% Die laatste trekt een lijn naar neuroimaging: super cool.
% Deep learning schept nieuwe mogelijkheden: dezelfde stimuli kunnen nu gebruikt worden.
% Coole resultaten wat betreft object recognition.
% Dus in deze studie proberen we een deep learning model van reading te maken.
% En we vergelijken de activatie in het model met MEG activatie tijdens lezen.
